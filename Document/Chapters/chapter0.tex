\chapter{Introducción}

\section{Análisis de sentimiento en Twitter en torno a política}


En el ámbito de la extracción de opiniones en el texto, Twitter ha sido analizado como un medio particularmente interesante para el estudio indirecto de fenómenos sociales. Esto se debe a su capacidad para captar a una gran cantidad de usuarios que discuten sobre una amplia variedad de tópicos, empleando una cantidad limitada de palabras y un estilo propio. 

Este enfoque ha sido estudiado en el pasado por \cite{o2010tweets}, quienes analizaron cómo los textos provenientes de esta plataforma coinciden con otras aproximaciones de la realidad social, como las encuestas de opinión, en relación a eventos de relevancia popular. De manera similar, \cite{tumasjan2010predicting} y \cite{mohammad2015sentiment} han utilizado esta red social para identificar las emociones presentes en los tweets relacionados con temas políticos. Cabe resaltar sin embargo que existe cierto sesgo en este tipo de estudios, pues la muestra de la población total que esta disponible en Twitter no es del todo representativa, puesta supone condiciones como el acceso a Internet que no son universales para toda la población.

No obstante, en el caso del español existen pocos estudios que se centren en la detección de emociones en redes sociales. Hasta ahora no se tiene conocimiento de ningún estudio que haya empleado modelos de lenguaje basados en Transformers con este propósito, especialmente en el contexto político 


\section{Avances técnicos recientes en análisis de sentimiento}

El estudio del análisis de sentimientos en general, y de emociones en particular, se ha realizado a través de técnicas tradicionales de NLP, cuyo enfoque se centra en la relación entre los términos y los estados emocionales. Recientemente, se ha avanzado hacia el uso de algoritmos basados en redes neuronales para capturar un sentido más general del texto basado en la relación que existe entre las palabras que lo componen. Inicialmente, para este propósito se utilizaron las RNN (Redes Neuronales Recurrentes), pero debido a su arquitectura, no era posible suministrar grandes cantidades de texto y captar la información contextual, ya que su entrenamiento se volvía demasiado pesado computacionalmente y la información del inicio del texto era difícilmente asociada a la información del final.

En ese contexto, surge una nueva arquitectura conocida como Transformers \citep{vaswani2017attention}, que elimina la estructura recurrente. Este modelo utiliza únicamente múltiples capas de auto-atención. Al eliminar los pasos recurrentes, permitió la paralelización del cálculo y poder entrenar la red con grandes volúmenes de texto de manera eficiente. Esto permite captar información del sentido del texto de una manera más compleja y computacionalmente más eficiente.

BERT \citep{devlin2018bert}, es un ejemplo de un modelo basado en Transformers que aprende el sentido del lenguaje de manera general a partir del entrenamiento con grandes volúmenes de datos provenientes de internet, como la Wikipedia. Este modelo pre-entrenado luego se utiliza para diversas tareas de procesamiento de lenguaje natural mediante el fine tuning de la última capa de la red, como por ejemplo el análisis de sentimiento. En el caso del español, existen distintos modelos basados en BERT específicamente para este idioma, que han sido entrenados a partir de textos provenientes de datos abiertos como el propuesto por \cite{canete2020spanish}. 





\section{Aporte de este trabajo}

El presente trabajo tiene como objetivo el empleo de modelos de lenguaje, específicamente modelos pre entrenados de lenguaje, para detectar emociones presentes en Twitter durante las elecciones presidenciales en Colombia en 2022. Estas consistieron en una primera vuelta celebrada el 29 de mayo y una segunda vuelta celebrada el 19 de junio. Los candidatos de la primera vuelta fueron Gustavo Petro, Rodolfo Hernandez, Federico Gutierres y Sergio Fajardo. El candidato Gustavo Petro es asociado a la izquierda Colombiana, Rodolo Hernandez es asociado a una derecha no tradicional, Federico Gutierrez se asocia a la derecha tradicional y Sergio Fajardo se asocia con el centro politico. Los ganadores de la primera vuelta fueron Gustavo petro con un 40.34\% y Rodolfo Hernandez con un 28.17\% pasando de este modo a la segunda vuelta. En esta, el ganador fue gustavo con un 50.44\% obteniendo Rodolfo Hernanez un 47.21\% \footnote{\url{https://es.wikipedia.org/wiki/Elecciones_presidenciales_de_Colombia_de_2022}}.

Para obtener los datos de entrenamiento, se utilizó una base de datos obtenida a través de la descarga de tweets mediante la API, utilizando hashtags relacionados con el tema de las elecciones, entre el 22 de mayo y el 22 de junio de 2022, período que comprende la primera y segunda vuelta presidencial de las elecciones en Colombia. Estos hashtags y por lo tanto, los tweets relacionados con ellos, fueron asociados por el autor a los sectores políticos de izquierda, derecha o neutro, basándose en el contenido observado en los tweets y en el apoyo o rechazo que estos muestren. A partir de ahí, se procedió a tomar una muestra de 1200 tweets para ser etiquetados manualmente por el autor y los directores, con alguna de las emociones disponibles. Con los tweets etiquetados, se procedió a realizar el fine tuning de varios modelos de lenguaje preentrenados y a evaluar su desempeño para seleccionar el mejor. Este modelo fue utilizado para clasificar toda la base de datos y a partir de esta clasificación, se realizaron agrupaciones que permitieron determinar la respuesta emocional de los tweets asociados a cierto sector político, así como la variación temporal de esta respuesta.


Se observo que las emociones asco y alegría tuvieron una presencia mayor en los tweets etiquetados manualmente con respecto a las emociones tristeza y miedo. Ademas, usualmente los tweets a los que se les era asignado tristeza o miedo, se les asignaba también asco. Esto provoco  un desempeño mas pobre con respecto a asco y alegría. En cuanto a los resultados de clasificación obtenidos por el modelo de mejor desempeño, robertuito, las etiquetas asignadas a los tweets de la base de datos,  coincidieron con las respuestas emocionales esperadas en eventos clave durante las elecciones. Se observó ademas una mayor presencia de la etiqueta alegría en tweets cuya orientación política asignada fue la  izquierda y más asco en los tweets a los que se les asigno orientación de  derecha. Este fenómeno se explica en parte por el hecho de que el candidato victorioso es de tendencia izquierdista. Ponemos a disposición el conjunto de datos y el modelo para futuras investigaciones sobre emociones en español en diferentes contextos.


