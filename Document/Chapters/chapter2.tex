\chapter{Datos}

En \cite{pak2010twitter} los autores utilizan distintos emoticones en la query de la API para traer tweets que tengan sentimientos positivos y negativos. 

En \cite{davidov2010enhanced} se utiliza un dataset de 450 millones de tweets se hace calcula la frecuencia de los hashtags hasta encontrar los mas comunes y de estos se seleccionan aquellos que tengan una asociación mas fuerte con los sentimientos para luego raer los tweets que los contengan.


En \cite{wang2012harnessing} se utilizan distintos términos relacionados con emociones que serán usados como hashtags para hacer el llamado a la API. luego se hace una preseleccion de los tweets en donde se verifique el hashtag este al final, que no contenga URL, que tenga mas de 5 palabras y que no tenga mas de 3 hashtags. Ademas, se agregaron variaciones lingüísticas a los términos: esperanza, esperanzador. inicialmente eran 5 millones de tweets que quedaron reducidos a 2 y medio


COn el objetyivo de contar con un dataset lo suficientemente robusto para la tarea, una vez descargados los tweets, estos fueron filtrados utilizando un modelo de lenguaje. Este modelo, fue entrenado con los datos provenientes de \cite{mohammad2018semeval}, \cite{sidorov2016construccion}, \cite{plaza2020emoevent}



\begin{table}[ht]
	\centering
	\caption{Cantidad de Tweets para cada emoción por Dataset}
\begin{tabular}{lrrrrrrr}
	\toprule
	{} &   ira &  miedo &  tristeza &  alegria &  asco &  sorpresa &  otra \\
	\midrule
	EmoEvent &   840 &     95 &       992 &     1756 &   160 &       340 &  4010 \\
	SemEval  &   914 &    396 &       849 &     1741 &   168 &       138 &  1016 \\
	Sidorov  &  1830 &   1873 &       282 &     1952 &  1862 &      1976 &     0 \\
	\bottomrule
\end{tabular}
\label{table:tweets_counts}
\end{table}