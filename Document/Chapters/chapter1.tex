\chapter{Introducción}






\section{Motivación}

Este trabajo es importante por que 



\section{Marco Teórico}

En \cite{ekman1993facial} Ekman habla sobre las seis emociones básicas que sirven de base para el estudio.

En \cite{ortony1987referential} se hace una relerencia al relacion que existe entre los estados emocionales y el lexico utilizado.

En \cite{strapparava2004wordnet} y \cite{esuli2006sentiwordnet} se expande este concepto para elaborar un léxico robusto asociado a emociones.

En \cite{wilson2009recognizing} se muestra como el contexto de na frase puede camiar el sentimiento de una palabra en particular

En el libro \cite{picard2000affective} Picard da un vistazo general sobre el uso de computadoras para detectar emociones.

En \cite{pang2002thumbs}, \cite{pang2004sentimental}, \cite{dave2003mining}, \cite{wilson2005recognizing}, \cite{turney2002thumbs}, \cite{nasukawa2003sentiment} y  se analiza la detección de sentimiento en el texto

En \cite{wiebe2005annotating}\cite{strapparava2008learning}, \cite{strapparava2007semeval}, \cite{alm2005emotions},\cite{aman2007identifying}, \cite{liu2003model}  se puede apreciar como el texto puede ser utilizado para detectar emociones.

Luego, en \cite{pang2008opinion} se muestra como los foros de Internet son una fuente de información de l cual se puede extraer valiosa información, entre esos detectar emociones.

En \cite{read2005using} se utilizan emoticones en los blogs de internet para detectar sentimiento.

En \cite{pak2010twitter}, \cite{kouloumpis2011twitter} y en \cite{go2009twitter}, \cite{barbosa2010robust} se aprecia como twitter puede ser usado como fuente para identificar sentimientos positivos, negativos y neutros.

En \cite{o2010tweets} se muestra como los sentimientos encontrados en twitter corresponden con resultados de encuestas de opinión.

En \cite{davidov2010enhanced} se utilizan los hashtags y los emoticones para la clasificacion


En \cite{hasan2014emotex}, , \cite{wang2012harnessing}y en \cite{roberts2012empatweet} se plantea la clasificación mediante distintos algoritmos de las emociones en los tweets.

En \cite{mohammad2012emotional} se hace uso de los hashtags para identificar emociones y entrenar los modelos.

En \cite{bollen2011modeling} se observa la relación entre los eventos sociales, políticos y económicos y las emociones detectadas.

En \cite{tumasjan2010predicting} se realiza un análisis de sentimientos durante una campaña política.






