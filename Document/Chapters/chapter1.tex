\chapter{Introducción}






\section{Motivación}

Este trabajo es importante por que 



\section{Marco Teórico}

En \cite{ekman1993facial} Ekman realiza un estudio de la respuesta fisiológica en general y de las  expresiones faciales del ser humano en diferentes culturas ante distintas circunstancias. Esto lo lleva a concluir que existen grandes grupos en donde las distintas expresiones faciales pueden ser agrupadas ya que estas reflejan el estado emocional interno de los individuos. A estos grupos los denomino emociones básicas y son los siguientes: Alegría, enojo, sorpresa, asco, miedo y tristeza. Este modelo de emociones básicas es comúnmente usado para los estudios relacionados con las emociones.

En el libro \cite{picard2000affective} Picard expone su punto de vista en donde parte de recientes investigaciones en psicología cognitiva que exponen las emociones como un componente fundamental de la inteligencia humana, por lo que la· búsqueda de una inteligencia artificial capaz de interactuar eficazmente con los seres humanos, debe traer consigo la capacidad de reconocer, entender, tener y expresar emociones. A partir de ahí expone los avances hechos hasta la fecha en el terreno de lo que ella llama, computación afectiva.

En \cite{ortony1987referential} se habla de como a pesar de que las emociones son procesos mentales internos que no residen en el lenguaje, este es el medio no fenomenológico mas conveniente a través del cual podemos acceder a ellas. Elabora entonces una serie condiciones que deben estar presentes en los términos para poder referirse de una manera acertada a los estados emocionales, y las aplica sobre términos presentes en la literatura con respecto a las emociones, construyendo así un léxico emocional.


En \cite{hatzivassiloglou1997predicting} a partir de un corpus extenso de adjetivos que vienen en parejas con usando distintos conectores, y un etiquetado manual de algunos de ellos, se establece un algoritmo para determinar lo que los autores denominan la orientación semántica de los mismos, esto es determinar si determinado adjetivo tiene una connotación negativa o positiva de la característica que describe. Este método permite la creación automática de un corpus extenso de adjetivos cuyo sentimiento se encuentra identificado.

En \cite{strapparava2004wordnet} se realiza una anotación manual de estados emocionales basados en las categorías de \cite{ortony1987referential} sobre algunos términos encontrados en WordNet \cite{miller1995wordnet}, que es una base de datos de términos en ingles agrupados por grupos de sinónimos y con relación semántica entre grupos. A partir de ahí se establece la categoría emocional de nuevos términos gracias a los sinónimos y las relaciones semánticas, construyendo así una base de datos de estados emocionales llamada WordNet-Affect

En \cite{wiebe1994tracking} se pone de manifiesto que en un tipo particular de texto, la ficción, la narración puede ocurrir desde un punto de vista objetivo,m es decir, una descripción de hechos comprobables, y también desde un punto de vista subjetivo, es decir, poniendo de manifiesto los hechos atravesados por los estados mentales internos el narrador, y que la distinción entre un tipo de narración y otra no esta siempre clara, por lo que propone un algoritmo capaz de hacer esta distinción de manera automática.


En \cite{yu2003towards} se plantea la necesidad de sistemas algoritmos capaces de discernir opiniones de hechos con el objetivo de lograr sistemas capaces de responde preguntas. Esta necesidad se plante a nivel de documentos, por ejemplo diferenciar editoriales de artículos de noticias, así como a nivel de frases. se plantean distintos modelos de aprendizaje supervisado para estas tareas en particular.



En \cite{pang2002thumbs} se establece la importancia de desarrollar, ante textos que reflejen opiniones subjetivas, sistemas capaces de identificar si dicha opinión es negativa o positiva. Para este propósito, se emplea el dominio de las reviews online de películas, construyendo algoritmos de aprendizaje supervisado que utiliza como features principalmente unigramas.

En \cite{turney2002thumbs} se pretende determinar a traves de aprendizaje no supervisado, si determinada review sobre diversos temas online, presenta un sentimiento negativo o positivo. Para ello, emplea el concepto de orientación semántica presente en \cite{hatzivassiloglou1997predicting} para determinar si una frase tiene orientación negativa o positiva para luego determinar si la review en su conjunto es positiva o negativa.

En \cite{wiebe2005annotating} se elabora una anotación manual de las estados emocionales presentes en las oraciones de un gran volumen de noticias, en donde se tiene en cuenta el contexto. El objetivo de esta anotación es crear un corpus etiquetado lo suficientemente grande para generar avances en el campo de la detección de emociones

En \cite{alm2005emotions} se utilizan los cuentos infantiles para desarrollar un modelo de aprendizaje supervisado capaz de detectar emociones en las frases del texto. Para ello se elabora una anotación manual de las frases que constituyen el set de datos y luego, se generan un set de features para estas que pasaran a entrenar un clasificador lineal.

En \cite{aman2007identifying} se utilizan texto proveniente de blogs para realizar detección de emociones presentes en las oraciones de estos. Para ello recurren primero a una anotación manual de las mismas y luego a la construcción de features para entrenar distintos modelos supervisados.


Luego, en \cite{pang2008opinion} se hace manifiesto la importancia que ha venido ganando el campo del análisis de sentimiento debido al auge del Internet, tanto para usuarios individuales como para la industria de la publicidad, el mercado financiero y la academia, por lo que se hace un recuento de las distintas técnicas y aplicaciones que son consideradas relevantes por los autores hasta la fecha. 




En \cite{pak2010twitter} se identifica twitter como una plataforma útil para extraer texto de diferentes usuarios sobre distintos temas por lo que se propone realizar un análisis de sentimientos sobre la misma. Para ello se procede  a la identificación de tweets que contengan emoticones felices y tristes, así como tweets provenientes de cuentas de medios de noticias para tweets neutrales. luego se procede a la construcción de features usando n-gramas a partir de las palabras presentes en el tweets  con estos se entrenaron varios clasificadores.



En \cite{o2010tweets} se plantea la pregunta si existe una correlación entre el sentimiento encontrado en twitter y las encuestas de opinión. Para ello, se toma una muestra del de mil millones de tweets entre 2008 y 2009 y se toman aquellos tweets que contengan palabras claves asociadas a los temas que se están investigando. EL sentimiento se determina a partir de la proporción de palabras con asociación negativa o positiva presentes en el tema que se esta analizando en un día en particular. Los resultados dan una correlación alta entre el sentimiento encontrado a través del texto y las encuestas.

En \cite{davidov2010enhanced} se parte del supuesto de que los hashtags y los emoticones contienen información relevante el cuanto al sentimiento del tweet, por lo que se hace una selección de 50 hashtags que tengan una asociación fuerte al sentimiento y se entrena un modelo supervisado a partir de los tweets que contengan estos hashtags, construyendo un vector de features para cada uno. El modelo es posteriormente empleado para clasificar otros tweets y jueces humanos verifican su eficacia.


En  \cite{wang2012harnessing} se entrena un modelo de supervisado distante de emociones presentes en twitter. Para ellos, se utilizan hashtags que contengan terminos claves provenientes de las 5 emociones basicas de propuestas por \cite{ekman1993facial} para realizar el llamado de la api. A partir de ahi se entrena el modelo a partir de los features construidos para el texto de los tweets. 

En \cite{roberts2012empatweet} se identifica la necesidad de contar con un corpus que sirva de base para la tarea de identificación de emociones en twitter. Para ello, se seleccionan  14 temas que para los autores tienen un fuerte contenido emocional y las palabras clave asociados a estos para ser usados como hashtags en las extracción. A partir de ahí, manualmente se etiquetaron los tweets con su respectiva emoción. Esto sirvió de base para entrenar un modelo de aprendizaje supervisado.



En \cite{bollen2011modeling} se plantea que a través del uso de twitter, los usurarios reflejan sus estados emocionales, ya sea de una manera explicita al indicar su emoción o de una manera implícita al hablar sobre un tema de interés general. En ese contexto, se genera procede a realizar una medida del estado emocional de una muestra de tweets entre agosto y diciembre del 2008, en donde el mismo se mide a través de ka similaridad presente entre las palabras de los tweets y ciertos términos claves asociados a estados emocionales. Esto permite encontrar que determinados eventos relevantes generan un impacto emocional significativo y durante un periodo de tiempo en los usuarios.

En \cite{tumasjan2010predicting} se plante el uso de twitter como plataforma de medición de la sensación política durante las elecciones parlamentarias en Alemania en el 2009. Una de sus preguntas de investigación estuvo relacionada con los sentimientos que se reflejan en los tweets que mencionan a los políticos que hacen parte de las campañas y para esto, se empleo sobre el texto proveniente de twitter, un software capaz de identificar palabras claves asociadas a estados emocionales y cognitivos en el texto. El resultado fue un perfil emocional para cada político que ve de acuerdo con su discurso político.

En \cite{ceron2016sentiment} se pretende realizar un análisis de sentimientos sobre tweets relacionados con las elecciones presidenciales en Colombia en el 2014 y comparar los resultados de este con las encuestas de opinión. Para ellos, inicialmente obtiene tweets que tengan  palabras claves y hashtags relacionados con las elecciones, luego hace un filtrado de spam y finalmente entrena un modelo supervisado con los features que genera para los tweets restantes. Los resultados no son consistentes por lo que se plantea en futuros trabajos  una caracterización demográfica.



Las RNN (redes neuronales recurrentes) son un tipo de arquitectura de red neuronal cuyo uso es especial para datos secuenciales, tales como las tareas de NLP. En estas, para un ejemplo nuevo, es posible utilizar el resultado del procesamiento de un dato anterior, para el procesamiento del siguiente dato (como en una secuencia de caracteres por ejemplo), es decir la misma topología de pesos recibirá para cada ejemplo nuevo, el resultado del ejemplo anterior ademas del ejemplo nuevo de entrada. Sin embargo, debido a su arquitectura, la optimizacion de los pesos que constituyen la topología se vuelve compleja pues para el calculo del gradiente del error, se deberá pasar tantas veces por los pesos de la red como pasos en el tiempo haya (como numero de caracteres en una palabra) haciendo que en secuencias particularmente largas, este calculo crezca o disminuya en demasía. Para enfrentar este problema , \cite{hochreiter1997long} y \cite{chung2014empirical} plantean nuevas arquitecturas de RNN conocidas como LSTM (Long-Short term memoryzz) y GRU (Gated Recurrent Unit) respectivamente, en donde a través de nuevas unidades que permiten la activación/cancelación de las señales que constituyen la red, se puede realizar la optimizacion de manera directa sin pasar por los pesos de la red.


En general ,las arquitecturas de RNN, al tener una estructura secuencial, se impide la paralelizacion de su computo, pues se necesita las salidas de ejemplos anteriores para llevar a cabo el siguiente paso en el tiempo. Ademas, debido a esta mismo funcionamiento secuencial, si existe un gran numero de pasos en el tiempo, es poco probable que la red tenga en cuenta información presente al inicio. Para solucionar estos inconvenientes, \cite{vaswani2017attention} desarrollan los Transformers, que son un tipo de arquitectura en la que todos los datos de entrada, son ingresados a la red de manera simultanea, como todas las palabras en una frase por ejemplo, y a través de una arquitectura que definen como atención, se realiza una transformación de los datos de manera que la representación de cada dato, en este caso cada palabra, tenga en cuenta que tan importante es la misma para todas las demás palabras de la oración. Esta transformación, aplicada tanto en los datos de entrada como en los de salida, es luego usada para entrenar los pesos de la red. De esta manera la red es capaz de procesar datos en simultaneo así como tener en cuenta toda la información disponible.
 

En \cite{devlin2018bert} se desarrolla BERT

En \cite{acheampong2021transformer} se hace un recuento de el uso de transformers para detectar emociones

EN \cite{canete2020spanish} se propone una aplicación de BERT para español

En \cite{gonzalez2021twilbert}, \cite{huang2019ana} se utiliza bert en twitter para  detectar emociones

En \cite{plaza2020improved}, \cite{gil2013combining} se hace una clasificación de emociones en español.


En \cite{sidorov2012empirical} propone un léxico de palabras en español asociadas a emociones
`








