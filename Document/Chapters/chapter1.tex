\chapter{Introducción}






\section{Motivación}

Este trabajo es importante por que 



\section{Marco Teórico}

En \cite{ekman1993facial} Ekman habla sobre las seis emociones básicas que sirven de base para el estudio

En el libro \cite{picard2000affective} Picard da un vistazo general sobre el uso de computadoras para detectar emociones.

En \cite{pang2002thumbs}, \cite{dave2003mining}, \cite{wilson2005recognizing}, \cite{alm2005emotions} se puede apreciar como el texto puede ser utilizado para detectar emociones.

Luego, en \cite{pang2008opinion} se muestra como los foros de internet son una fuente de información de l cual se puede extraer valiosa informacion, entre esos detectar emociones.

En \cite{pak2010twitter}, \cite{kouloumpis2011twitter} y en \cite{go2009twitter} se aprecia como twitter puede ser usado como fuente para identificar sentimientos positivos, negativos y neutros.

En \cite{hasan2014emotex}, \cite{bollen2011modeling}, \cite{strapparava2008learning}, \cite{strapparava2007semeval}, y en \cite{roberts2012empatweet} se plantea la clasificación mediante distintos algoritmos de las emociones en los tweets.





