\chapter{Introducción}






\section{Motivación}

Este trabajo es importante por que 



\section{Marco Teórico}

En \cite{ekman1993facial} Ekman realiza un estudio de la respuesta fisiológica en general y de las  expresiones faciales del ser humano en diferentes culturas ante distintas circunstancias. Esto lo lleva a concluir que existen grandes grupos en donde las distintas expresiones faciales pueden ser agrupadas ya que estas reflejan el estado emocional interno de los individuos. A estos grupos los denomino emociones básicas y son los siguientes: Alegría, enojo, sorpresa, asco, miedo y tristeza. Este modelo de emociones básicas es comúnmente usado para los estudios relacionados con las emociones.

En el libro \cite{picard2000affective} Picard expone su punto de vista en donde parte de recientes investigaciones en psicología cognitiva que exponen las emociones como un componente fundamental de la inteligencia humana, por lo que la· búsqueda de una inteligencia artificial capaz de interactuar eficazmente con los seres humanos, debe traer consigo la capacidad de reconocer, entender, tener y expresar emociones. A partir de ahí expone los avances hechos hasta la fecha en el terreno de lo que ella llama, computación afectiva.

En \cite{ortony1987referential} se habla de como a pesar de que las emociones son procesos mentales internos que no residen en el lenguaje, este es el medio no fenomenológico mas conveniente a través del cual podemos acceder a ellas. Elabora entonces una serie condiciones que deben estar presentes en los términos para poder referirse de una manera acertada a los estados emocionales, y las aplica sobre términos presentes en la literatura con respecto a las emociones, construyendo así un léxico emocional.


En \cite{hatzivassiloglou1997predicting} a partir de un corpus extenso de adjetivos que vienen en parejas con usando distintos conectores, y un etiquetado manual de algunos de ellos, se establece un algoritmo para determinar lo que los autores denominan la orientación semántica de los mismos, esto es determinar si determinado adjetivo tiene una connotación negativa o positiva de la característica que describe. Este método permite la creación automática de un corpus extenso de adjetivos cuyo sentimiento se encuentra identificado.

En \cite{strapparava2004wordnet} se realiza una anotación manual de estados emocionales basados en las categorías de \cite{ortony1987referential} sobre algunos términos encontrados en WordNet \cite{miller1995wordnet}, que es una base de datos de términos en ingles agrupados por grupos de sinónimos y con relación semántica entre grupos. A partir de ahí se establece la categoría emocional de nuevos términos gracias a los sinónimos y las relaciones semánticas, construyendo así una base de datos de estados emocionales llamada WordNet-Affect

En \cite{wiebe1994tracking} se pone de manifiesto que en un tipo particular de texto, la ficción, la narración puede ocurrir desde un punto de vista objetivo,m es decir, una descripción de hechos comprobables, y también desde un punto de vista subjetivo, es decir, poniendo de manifiesto los hechos atravesados por los estados mentales internos el narrador, y que la distinción entre un tipo de narración y otra no esta siempre clara, por lo que propone un algoritmo capaz de hacer esta distinción de manera automática.


En \cite{yu2003towards} se plantea la necesidad de sistemas algoritmos capaces de discernir opiniones de hechos con el objetivo de lograr sistemas capaces de responde preguntas. Esta necesidad se plante a nivel de documentos, por ejemplo diferenciar editoriales de artículos de noticias, así como a nivel de frases. se plantean distintos modelos de aprendizaje supervisado para estas tareas en particular.



En \cite{pang2002thumbs} se establece la importancia de desarrollar, ante textos que reflejen opiniones subjetivas, sistemas capaces de identificar si dicha opinión es negativa o positiva. Para este propósito, se emplea el dominio de las reviews online de películas, construyendo algoritmos de aprendizaje supervisado que utiliza como features principalmente unigramas.

En \cite{turney2002thumbs} se pretende determinar a traves de aprendizaje no supervisado, si determinada review sobre diversos temas online, presenta un sentimiento negativo o positivo. Para ello, emplea el concepto de orientación semántica presente en \cite{hatzivassiloglou1997predicting} para determinar si una frase tiene orientación negativa o positiva para luego determinar si la review en su conjunto es positiva o negativa.

En \cite{wiebe2005annotating} se elabora una anotación manual de las estados emocionales presentes en las oraciones de un gran volumen de noticias, en donde se tiene en cuenta el contexto. El objetivo de esta anotación es crear un corpus etiquetado lo suficientemente grande para generar avances en el campo de la detección de emociones

En \cite{alm2005emotions} se utilizan los cuentos infantiles para desarrollar un modelo de aprendizaje supervisado capaz de detectar emociones en las frases del texto. Para ello se elabora una anotación manual de las frases que constituyen el set de datos y luego, se generan un set de features para estas que pasaran a entrenar un clasificador lineal.

En \cite{aman2007identifying} se utilizan texto proveniente de blogs para realizar detección de emociones presentes en las oraciones de estos. Para ello recurren primero a una anotación manual de las mismas y luego a la construcción de features para entrenar distintos modelos supervisados.


Luego, en \cite{pang2008opinion} se hace manifiesto la importancia que ha venido ganando el campo del análisis de sentimiento debido al auge del Internet, tanto para usuarios individuales como para la industria de la publicidad, el mercado financiero y la academia, por lo que se hace un recuento de las distintas técnicas y aplicaciones que son consideradas relevantes por los autores hasta la fecha. 




En \cite{pak2010twitter} se identifica twitter como una plataforma útil para extraer texto de diferentes usuarios sobre distintos temas por lo que se propone realizar un análisis de sentimientos sobre la misma. Para ello se procede  a la identificación de tweets que contengan emoticones felices y tristes, así como tweets provenientes de cuentas de medios de noticias para tweets neutrales. luego se procede a la construcción de features usando n-gramas a partir de las palabras presentes en el tweets  con estos se entrenaron varios clasificadores.



En \cite{o2010tweets} se plantea la pregunta si existe una correlación entre el sentimiento encontrado en twitter y las encuestas de opinión. Para ello, se toma una muestra del de mil millones de tweets entre 2008 y 2009 y se toman aquellos tweets que contengan palabras claves asociadas a los temas que se están investigando. EL sentimiento se determina a partir de la proporción de palabras con asociación negativa o positiva presentes en el tema que se esta analizando en un día en particular. Los resultados dan una correlación alta entre el sentimiento encontrado a través del texto y las encuestas.

En \cite{davidov2010enhanced} se parte del supuesto de que los hashtags y los emoticones contienen información relevante el cuanto al sentimiento del tweet, por lo que se hace una selección de 50 hashtags que tengan una asociación fuerte al sentimiento y se entrena un modelo supervisado a partir de los tweets que contengan estos hashtags, construyendo un vector de features para cada uno. El modelo es posteriormente empleado para clasificar otros tweets y jueces humanos verifican su eficacia.


En  \cite{wang2012harnessing} se entrena un modelo de supervisado distante de emociones presentes en twitter. Para ellos, se utilizan hashtags que contengan terminos claves provenientes de las 5 emociones basicas de propuestas por \cite{ekman1993facial} para realizar el llamado de la api. A partir de ahi se entrena el modelo a partir de los features construidos para el texto de los tweets. 

En \cite{roberts2012empatweet} se identifica la necesidad de contar con un corpus que sirva de base para la tarea de identificación de emociones en twitter. Para ello, se seleccionan  14 temas que para los autores tienen un fuerte contenido emocional y las palabras clave asociados a estos para ser usados como hashtags en las extracción. A partir de ahí, manualmente se etiquetaron los tweets con su respectiva emoción. Esto sirvió de base para entrenar un modelo de aprendizaje supervisado.



En \cite{bollen2011modeling} se observa la relación entre los eventos sociales, políticos y económicos y las emociones detectadas.

En \cite{tumasjan2010predicting} se realiza un análisis de sentimientos durante una campaña política.

En \cite{ceron2016sentiment} explora el análisis de sentimiento en twitter durante las elecciones 2014 en Colombia.

En \cite{hochreiter1997long} se desarrollan las redes LSTM que son un tipo de RNN.

En \cite{chung2014empirical} se utilizan las GRU que son otro tipo de RNN y superan a las LSTM

En \cite{vaswani2017attention} se desarrollan la tecnica de trasnformers que combina LSTM y GRU, de las cuales bert es un ejemplo

En \cite{devlin2018bert} se desarrolla BERT

En \cite{acheampong2021transformer} se hace un recuento de el uso de transformers para detectar emociones

EN \cite{canete2020spanish} se propone una aplicación de BERT para español

En \cite{gonzalez2021twilbert}, \cite{huang2019ana} se utiliza bert en twitter para  detectar emociones

En \cite{plaza2020improved}, \cite{gil2013combining} se hace una clasificación de emociones en español.


En \cite{sidorov2012empirical} propone un léxico de palabras en español asociadas a emociones
`








