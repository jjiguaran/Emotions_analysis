\chapter{Introducción}






\section{Motivación}

Este trabajo es importante por que 



\section{Marco Teórico}

En \cite{ekman1993facial} Ekman realiza un estudio de la respuesta fisiológica en general y de las  expresiones faciales del ser humano en diferentes culturas ante distintas circunstancias. Esto lo lleva a concluir que existen grandes grupos en donde las distintas expresiones faciales pueden ser agrupadas ya que estas reflejan el estado emocional interno de los individuos. A estos grupos los denomino emociones básicas y son los siguientes: Alegría, enojo, sorpresa, asco, miedo y tristeza. Este modelo de emociones básicas es comúnmente usado para los estudios relacionados con las emociones.

En el libro \cite{picard2000affective} Picard expone su punto de vista en donde parte de recientes investigaciones en psicología cognitiva que exponen las emociones como un componente fundamental de la inteligencia humana, por lo que la· búsqueda de una inteligencia artificial capaz de interactuar eficazmente con los seres humanos, debe traer consigo la capacidad de reconocer, entender, tener y expresar emociones. A partir de ahí expone los avances hechos hasta la fecha en el terreno de lo que ella llama, computación afectiva.

En \cite{ortony1987referential} se habla de como a pesar de que las emociones son procesos mentales internos que no residen en el lenguaje, este es el medio no fenomenológico mas conveniente a través del cual podemos acceder a ellas. Elabora entonces una serie condiciones que deben estar presentes en los términos para poder referirse de una manera acertada a los estados emocionales, y las aplica sobre términos presentes en la literatura con respecto a las emociones, construyendo así un léxico emocional.


En \cite{hatzivassiloglou1997predicting} a partir de un corpus extenso de adjetivos que vienen en parejas con usando distintos conectores, y un etiquetado manual de algunos de ellos, se establece un algoritmo para determinar lo que los autores denominan la orientación semántica de los mismos, esto es determinar si determinado adjetivo tiene una connotación negativa o positiva de la característica que describe. Este método permite la creación automática de un corpus extenso de adjetivos cuyo sentimiento se encuentra identificado.

En \cite{strapparava2004wordnet} se realiza una anotación manual de estados emocionales basados en las categorias de \cite{ortony1987referential} sobre algunos términos encontrados en WordNet \cite{miller1995wordnet}, que es una base de datos de términos en ingles agrupados por grupos de sinónimos y con relación semántica entre grupos. A partir de ahí se establece la categoría emocional de nuevos términos gracias a los sinónimos y las relaciones semánticas, construyendo así una base de datos de estados emocionales llamada WordNet-Affect

En \cite{wiebe1994tracking} se plantea que el analisis de sentimiento es un caso particular de analisis de subjetividad.


En \cite{yu2003towards} se plantea un metodo para separar opiniones de hechos.





En \cite{pang2002thumbs}, \cite{turney2002thumbs},  y  se analiza la detección de sentimiento en el texto

En \cite{wiebe2005annotating}, \cite{alm2005emotions},\cite{aman2007identifying}  se puede apreciar como el texto puede ser utilizado para detectar emociones.

Luego, en \cite{pang2008opinion} se muestra como los foros de Internet son una fuente de información de l cual se puede extraer valiosa información, entre esos detectar emociones.


En \cite{pak2010twitter}, \cite{kouloumpis2011twitter} y en \cite{go2009twitter}e aprecia como twitter puede ser usado como fuente para identificar sentimientos positivos, negativos y neutros.

En \cite{o2010tweets} se muestra como los sentimientos encontrados en twitter corresponden con resultados de encuestas de opinión.

En \cite{davidov2010enhanced} se utilizan los hashtags y los emoticones para la clasificacion


En  , \cite{wang2012harnessing}y en \cite{roberts2012empatweet} se plantea la clasificación mediante distintos algoritmos de las emociones en los tweets.



En \cite{bollen2011modeling} se observa la relación entre los eventos sociales, políticos y económicos y las emociones detectadas.

En \cite{tumasjan2010predicting} se realiza un análisis de sentimientos durante una campaña política.

En \cite{ceron2016sentiment} explora el análisis de sentimiento en twitter durante las elecciones 2014 en Colombia.

En \cite{hochreiter1997long} se desarrollan las redes LSTM que son un tipo de RNN.

En \cite{chung2014empirical} se utilizan las GRU que son otro tipo de RNN y superan a las LSTM

En \cite{vaswani2017attention} se desarrollan la tecnica de trasnformers que combina LSTM y GRU, de las cuales bert es un ejemplo

En \cite{devlin2018bert} se desarrolla BERT

En \cite{acheampong2021transformer} se hace un recuento de el uso de transformers para detectar emociones

EN \cite{canete2020spanish} se propone una aplicación de BERT para español

En \cite{gonzalez2021twilbert}, \cite{huang2019ana} se utiliza bert en twitter para  detectar emociones

En \cite{plaza2020improved}, \cite{gil2013combining} se hace una clasificación de emociones en español.


En \cite{sidorov2012empirical} propone un léxico de palabras en español asociadas a emociones
`








