\chapter{Introducción}






\section{Motivación}

Este trabajo es importante por que 



\section{Marco Teórico}

En \cite{ekman1993facial} Ekman realiza un estudio de la respuesta fisiológica en general y de las  expresiones faciales del ser humano en diferentes culturas ante distintas circunstancias. Esto lo lleva a concluir que existen grandes grupos en donde las distintas expresiones faciales pueden ser agrupadas ya que estas reflejan el estado emocional interno de los individuos. A estos grupos los denomino emociones básicas y son los siguientes: Alegría, enojo, sorpresa, asco, miedo y tristeza. Este modelo de emociones básicas es comúnmente usado para los estudios relacionados con las emociones.

En el libro \cite{picard2000affective} Picard da un vistazo general sobre el uso de computadoras para detectar emociones.

En \cite{ortony1987referential} se hace una relerencia al relacion que existe entre los estados emocionales y el lexico utilizado.


En \cite{hatzivassiloglou1997predicting}, \cite{strapparava2004wordnet}  se expande este concepto para elaborar un léxico robusto asociado a emociones.

En \cite{wiebe1994tracking} se plantea que el analisis de sentimiento es un caso particular de analisis de subjetividad.


En \cite{yu2003towards} se plantea un metodo para separar opiniones de hechos.





En \cite{pang2002thumbs}, \cite{turney2002thumbs},  y  se analiza la detección de sentimiento en el texto

En \cite{wiebe2005annotating}, \cite{alm2005emotions},\cite{aman2007identifying}  se puede apreciar como el texto puede ser utilizado para detectar emociones.

Luego, en \cite{pang2008opinion} se muestra como los foros de Internet son una fuente de información de l cual se puede extraer valiosa información, entre esos detectar emociones.


En \cite{pak2010twitter}, \cite{kouloumpis2011twitter} y en \cite{go2009twitter}e aprecia como twitter puede ser usado como fuente para identificar sentimientos positivos, negativos y neutros.

En \cite{o2010tweets} se muestra como los sentimientos encontrados en twitter corresponden con resultados de encuestas de opinión.

En \cite{davidov2010enhanced} se utilizan los hashtags y los emoticones para la clasificacion


En  , \cite{wang2012harnessing}y en \cite{roberts2012empatweet} se plantea la clasificación mediante distintos algoritmos de las emociones en los tweets.



En \cite{bollen2011modeling} se observa la relación entre los eventos sociales, políticos y económicos y las emociones detectadas.

En \cite{tumasjan2010predicting} se realiza un análisis de sentimientos durante una campaña política.

En \cite{ceron2016sentiment} explora el análisis de sentimiento en twitter durante las elecciones 2014 en Colombia.

En \cite{hochreiter1997long} se desarrollan las redes LSTM que son un tipo de RNN.

En \cite{chung2014empirical} se utilizan las GRU que son otro tipo de RNN y superan a las LSTM

En \cite{vaswani2017attention} se desarrollan la tecnica de trasnformers que combina LSTM y GRU, de las cuales bert es un ejemplo

En \cite{devlin2018bert} se desarrolla BERT

En \cite{acheampong2021transformer} se hace un recuento de el uso de transformers para detectar emociones

EN \cite{canete2020spanish} se propone una aplicación de BERT para español

En \cite{gonzalez2021twilbert}, \cite{huang2019ana} se utiliza bert en twitter para  detectar emociones

En \cite{plaza2020improved}, \cite{gil2013combining} se hace una clasificación de emociones en español.


En \cite{sidorov2012empirical} propone un léxico de palabras en español asociadas a emociones
`








