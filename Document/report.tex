%%%%%%%%%%%%%%%%%%%%%%%%%%%%%%%%%%%%%%%%%
% IISER Thiruvananthapuram Thesis Report Format
% LaTeX Template
%
% Author:
% Nikhil Alex Verghese, BS-MS 17
% PLEASE FORWARD ANY AND ALL SUGGESTIONS AND COMPLAINTS TO: nikhil.alexv@gmail.com
%
% READ ALL INSTRUCTIONS IN EACH TEX FILE CAREFULLY, ESPECIALLY THE MAIN REPORT FILE.
%
% License:
% CC BY-NC-SA 4.0 (http://creativecommons.org/licenses/by-nc-sa/4.0/)
%
%%%%%%%%%%%%%%%%%%%%%%%%%%%%%%%%%%%%%%%%%

%----------------------------------------------------------------------------------------
%	PACKAGES AND OTHER DOCUMENT CONFIGURATIONS
%----------------------------------------------------------------------------------------

\documentclass[12pt,a4wide]{report} % Set the font size (10pt, 11pt and 12pt) and paper size (letterpaper, a4paper, etc)

\usepackage{amsthm,amssymb,mathrsfs,setspace,pstricks,booktabs,mathtools,amsmath,geometry}

% UNCOMMENT REQUIRED PACKAGES
%
% latexsym package contains few extra symbols.
% footmisc package for typesetting footnotes better.
% hypperref package for creating for reliable hyperlinks and customizations.
% tikz package for drawing graphs and diagrams (XY-pic is now outdated).
% xcolor package for better control over colours.
% biblatex package for bibliography-related (modern form of bibtex; biber and natbib also available, use only one).
% READ ALL ASSOCIATED PACKAGE HELP FILES CAREFULLY BEFORE USE.
%
% \usepackage{latexsym,footmisc,biblatex,hyperref,xcolor,tikz}.

\usepackage{play}
\usepackage{epsfig}
\usepackage[position=t,singlelinecheck=off]{subfig}


%\usepackage[grey,times]{quotchap}
\usepackage[nottoc]{tocbibind}
\renewcommand{\chaptermark}[1]{\markboth{#1}{}}
\renewcommand{\sectionmark}[1]{\markright{\thesection\ #1}}

\setlength{\parskip}{1em plus 0.25em minus 0.25em}

\theoremstyle{plain}
\newtheorem{theorem}{Theorem}[section]
\newtheorem{lemma}[theorem]{Lemma}
\newtheorem{corollary}[theorem]{Corollary}
\newtheorem{proposition}[theorem]{Proposition}

\theoremstyle{definition}
\newtheorem{definition}[theorem]{Definition}
\newtheorem{example}[theorem]{Example}
\newtheorem{notation}[theorem]{Notation}

\theoremstyle{remark}
\newtheorem{remark}[theorem]{Remark}

\renewcommand{\baselinestretch}{1.5}

% Uncomment below for headers and footers.
% Further customization codes can be referred here: https://www.overleaf.com/learn/latex/Headers_and_footers
%\usepackage{fancyhdr}
%\pagestyle{fancy}

% Uncomment for some standard notations in math (Real, Complex and Rational numbers, Norm, Jacobian, etc) %
%\newcommand{\reals}{\mathbb{R}}
%\newcommand{\complex}{\mathbb{C}}
%\newcommand{\rational}{\mathbb{Q}}
%\newcommand{\jacobian}{\mathcal{J}}
%\newcommand{\norm}[1]{\left\lVert #1 \right\rVert}

\begin{document}

% CUSTOM INPUT FIELDS ARE MARKED WITH [] IN THE INTRO SECTIONS.
% GO THROUGH EACH INTRO PAGE AND UPDATE ALL YOUR PERSONAL INFORMATION ACCORDINGLY.
% ALSO READ ALL UNCOMMENTED LINES CAREFULLY.

% --------------- Title page -----------------------%
\begin{titlepage}
\enlargethispage{3cm}

\begin{center}

\vspace*{-1cm}

\textbf{\Large Aplicación de modelos de lenguaje para la identificación de emociones presentes en twitter durante el periodo de elecciones presidenciales en Colombia 2022}\\[10pt]

\vspace*{0.5cm}


Tesis presentada para optar por el titulo de\\ 
\vspace*{0.5cm}
{\Large \bf Magister en Explotación de Datos y Descubrimiento del Conocimiento } \\





                      \vspace{10mm}
                   {\em  por} \\ \vspace{3mm}
             {\large \bf Juan Jose Iguaran Fernandez} \\


\vspace*{10mm}

\begin{figure}[h]
  \begin{center}
  \subfloat{{\includegraphics[scale=0.55]{Images & Logos/data_mining.png} }}%
  \subfloat{{\includegraphics[scale=0.18]{Images & Logos/uba.png}
   }}
  %
  \end{center}
\end{figure}

\vspace*{10mm}


{\bf\large Universidad de Buenos Aires} \\[10pt]
{\bf\large Facultad de Ciencias Exactas y Naturales}\\%[4pt]



{\bf\large Departamento de Ciencias de la Computación}\\%[8pt]
{\it\large [Insert Month and Year]}

\end{center}

\end{titlepage}

\clearpage


% --------------- Declaration page -----------------------%
\pagenumbering{roman} \setcounter{page}{2}
\begin{center}
{\Large{\bf{DECLARATION}}}
\end{center}

\noindent

I, \textbf{[Type Your Full Name] (Roll No: [Type your Roll Number])}, hereby declare that, this report entitled \textbf{``[Project Title]”} submitted to Indian Institute of Science Education and Research Thiruvananthapuram towards partial requirement of \textbf{Master of Science} in \textbf{[Type your Stream]} is an original work carried out by me under the supervision of [Project Supervisor] and has not formed the basis for the award of any degree or diploma, in this or any other institution or university. I have sincerely tried to uphold the academic ethics and honesty. Whenever an external information or statement or result is used then, that have been duly acknowledged and cited.

\vspace{4cm} %Reduce if text overflowing to a new page

\noindent Thiruvananthapuram - 695 551 \hfill \textbf{[Type Your Name]}

\noindent [Insert Month and Year] \hfill

\clearpage

% --------------- Certificate page -----------------------%
\begin{center}
{\large{\bf{CERTIFICATE}}}
\end{center}
%\thispagestyle{empty}


\noindent
This is to certify that the work contained in this project report entitled \textbf{``[Project Title]''}  submitted by \textbf{[Your Name]} (\textbf{Roll No: [Your Roll Number]}) to Indian Institute of Science Education and Research, Thiruvananthapuram towards the partial requirement of {\bf [Master of Science/ Doctor of Philosophy]} in \textbf{[Branch of Science]} has been carried out by [him/her/them] under my supervision and that it has not been submitted elsewhere for the award of any degree.


\vspace{4cm} %Reduce if text overflowing to a new page

\noindent Thiruvananthapuram - 695 551 \hfill [Project Supervisor Name]

\noindent [Insert Month and Year] \hfill Project Supervisor

\clearpage


% Switch from 03a_Certificate (line 81) to 03b_Certificate (line 84) for a fancier format.
% You can disable geometry package if not switching and compile time is too long. 
% \thispagestyle{plain}
\newgeometry{left=1cm,top=2cm,right=1cm}


 \flushleft
 \includegraphics[width=40mm]{Images & Logos/iiser_logo.png}

\vspace{0.5\baselineskip}
\hrule
\vspace{3\baselineskip}

\begin{center}
{\Large {\bf Certificate}}
\end{center}

\vspace{\baselineskip}

\noindent This is to certify that the work contained in this project report entitled
"\textbf{[Insert Project Title]}" submitted by \textbf{[Insert Name]} (Roll No. \textbf{[Insert Roll Number]}) to the Indian Institute of Science Education and Research, Thiruvananthapuram towards the partial requirement of {\bf [Master of Science/ Doctor of Philosophy]} in \textbf{[Branch of Science]} has been carried out by [him/her/them] under my supervision and that it has not been submitted elsewhere for the award of any degree.

\vspace{3\baselineskip}
\begin{flushright}
\begin{minipage}[c]{0.45\textwidth}
\centering
\vspace{3\baselineskip}
\hrule
\vspace{1.5\baselineskip}
{\large [Project Supervisor Name]} \bigskip\\
{\large \bf Project Supervisor} \\
\large [Insert Department] ~\\\
IISER Thiruvananthapuram
\end{minipage}
\end{flushright}
\vspace{\baselineskip}
\restoregeometry

% Credit for Certificate code file: Sagnik Saha, IISER B'16
% ----------------- Acknowledgement page--------------------%
\begin{center}
{\large{\bf{ACKNOWLEDGEMENT}}}
\end{center}
%\thispagestyle{empty}


\noindent
[\textit{Sample:}] I want to extend a sincere and heartfelt obligation towards all the personages without whom the completion of the project was not possible. I express my profound gratitude and deep regard to [Name of Professor], IISER Thiruvananthapuram for [his/her/their] guidance, valuable feedback, and constant encouragement throughout the project. [His/Her/Their] valuable suggestions were of immense help. I sincerely acknowledge [his/her/their] constant support and guidance during the project. 

I am immensely grateful to [Insert Names] for their constant support and encouragement. I am also grateful to the Indian Institute of Science Education and Research, Thiruvananthapuram, for allowing me to do this project and providing all the required facilities.

% Use words like "hard work","helping every step of the way", 
% "sincere gratitude", "deepest appreciation", "highly indebted", "considerate endorsement", "honest and cooperative response", etc.
% "constant support, utmost patience and trust with respect to this project"
% "intuitiveness and insight has been invaluable to the progression of this project, allowing it to mature into the project it is today."
% "valuable guidance kept the project afloat especially with [his/her/their] fresh take with every stage of development of this project."
%"show my deepest appreciation towards my close friends and family for their pivotal care and well wishes and for encouraging me every step of the way, including but not limited to"
% Credit for Sample: Prem Ghonmode, IISER B'17

% Ctrl+b for \textbf{} and Ctrl+i for \textit{}


\vspace{4cm} %Reduce if text overflowing to a new page

\noindent Thiruvananthapuram - 695 551 \hfill \textbf{[Type Your Name]}

\noindent [Insert Month and Year] \hfill

\clearpage
% -------------------- Abstract page -----------------------%
\begin{center}
{\Large{\bf{Resumen}}}
\end{center}


El estudio de Twitter como medio para el análisis de fenómenos sociales mediante técnicas de procesamiento del lenguaje natural (NLP) es de gran interés. Dentro de estas técnicas, la detección de emociones en el texto es especialmente relevante , ya que permite identificar la respuesta subjetiva de las personas ante los eventos sociales. Los modelos de lenguaje basados en redes neuronales, como los Transformers y específicamente BERT, han reemplazado las técnicas tradicionales de NLP debido a su capacidad para capturar el contexto y las relaciones entre las palabras en el texto. Sin embargo, en español, hay pocos estudios de detección de emociones en redes sociales y ninguno que utilice modelos de lenguaje basados en Transformers en un contexto político. Este trabajo tiene como objetivo emplear modelos basados en BERT para detectar emociones en Twitter durante las elecciones presidenciales de Colombia en 2022. Se utilizará una base de datos de tweets descargados a través de la API, etiquetados manualmente con emociones y se realizará un fine tuning de varios modelos de lenguaje preentrenados para seleccionar el mejor. Este modelo se utilizará para clasificar toda la base de datos y analizar la respuesta emocional de los tweets asociados a sectores políticos y su variación temporal.

\vspace{4cm} % Reduce if text overflowing to a new page. Don't make it too long.
\textbf{Palabras Clave:} [BERT, Colombia, Elecciones, Detección de emociones]

\clearpage

% ------------------ Table of Contents --------------------%

\tableofcontents
\clearpage
\listoffigures
\listoftables

\newpage

\pagenumbering{arabic}
\setcounter{page}{1}

% =========== Main chapters starts here =================== %

\chapter{Introducción}






\section{Motivación}

Este trabajo es importante por que 



\section{Marco Teórico}

En \cite{ekman1993facial} Ekman realiza un estudio de la respuesta fisiológica en general y de las  expresiones faciales del ser humano en diferentes culturas ante distintas circunstancias. Esto lo lleva a concluir que existen grandes grupos en donde las distintas expresiones faciales pueden ser agrupadas ya que estas reflejan el estado emocional interno de los individuos. A estos grupos los denomino emociones básicas y son los siguientes: Alegría, enojo, sorpresa, asco, miedo y tristeza. Este modelo de emociones básicas es comúnmente usado para los estudios relacionados con las emociones.

En el libro \cite{picard2000affective} Picard expone su punto de vista en donde parte de recientes investigaciones en psicología cognitiva que exponen las emociones como un componente fundamental de la inteligencia humana, por lo que la· búsqueda de una inteligencia artificial capaz de interactuar eficazmente con los seres humanos, debe traer consigo la capacidad de reconocer, entender, tener y expresar emociones. A partir de ahí expone los avances hechos hasta la fecha en el terreno de lo que ella llama, computación afectiva.

En \cite{ortony1987referential} se habla de como a pesar de que las emociones son procesos mentales internos que no residen en el lenguaje, este es el medio no fenomenológico mas conveniente a través del cual podemos acceder a ellas. Elabora entonces una serie condiciones que deben estar presentes en los términos para poder referirse de una manera acertada a los estados emocionales, y las aplica sobre términos presentes en la literatura con respecto a las emociones, construyendo así un léxico emocional.


En \cite{hatzivassiloglou1997predicting} a partir de un corpus extenso de adjetivos que vienen en parejas con usando distintos conectores, y un etiquetado manual de algunos de ellos, se establece un algoritmo para determinar lo que los autores denominan la orientación semántica de los mismos, esto es determinar si determinado adjetivo tiene una connotación negativa o positiva de la característica que describe. Este método permite la creación automática de un corpus extenso de adjetivos cuyo sentimiento se encuentra identificado.

En \cite{strapparava2004wordnet} se realiza una anotación manual de estados emocionales basados en las categorías de \cite{ortony1987referential} sobre algunos términos encontrados en WordNet \cite{miller1995wordnet}, que es una base de datos de términos en ingles agrupados por grupos de sinónimos y con relación semántica entre grupos. A partir de ahí se establece la categoría emocional de nuevos términos gracias a los sinónimos y las relaciones semánticas, construyendo así una base de datos de estados emocionales llamada WordNet-Affect

En \cite{wiebe1994tracking} se pone de manifiesto que en un tipo particular de texto, la ficción, la narración puede ocurrir desde un punto de vista objetivo,m es decir, una descripción de hechos comprobables, y también desde un punto de vista subjetivo, es decir, poniendo de manifiesto los hechos atravesados por los estados mentales internos el narrador, y que la distinción entre un tipo de narración y otra no esta siempre clara, por lo que propone un algoritmo capaz de hacer esta distinción de manera automática.


En \cite{yu2003towards} se plantea la necesidad de sistemas algoritmos capaces de discernir opiniones de hechos con el objetivo de lograr sistemas capaces de responde preguntas. Esta necesidad se plante a nivel de documentos, por ejemplo diferenciar editoriales de artículos de noticias, así como a nivel de frases. se plantean distintos modelos de aprendizaje supervisado para estas tareas en particular.



En \cite{pang2002thumbs} se establece la importancia de desarrollar, ante textos que reflejen opiniones subjetivas, sistemas capaces de identificar si dicha opinión es negativa o positiva. Para este propósito, se emplea el dominio de las reviews online de películas, construyendo algoritmos de aprendizaje supervisado que utiliza como features principalmente unigramas.

En \cite{turney2002thumbs} se pretende determinar a traves de aprendizaje no supervisado, si determinada review sobre diversos temas online, presenta un sentimiento negativo o positivo. Para ello, emplea el concepto de orientación semántica presente en \cite{hatzivassiloglou1997predicting} para determinar si una frase tiene orientación negativa o positiva para luego determinar si la review en su conjunto es positiva o negativa.

En \cite{wiebe2005annotating} se elabora una anotación manual de las estados emocionales presentes en las oraciones de un gran volumen de noticias, en donde se tiene en cuenta el contexto. El objetivo de esta anotación es crear un corpus etiquetado lo suficientemente grande para generar avances en el campo de la detección de emociones

En \cite{alm2005emotions} se utilizan los cuentos infantiles para desarrollar un modelo de aprendizaje supervisado capaz de detectar emociones en las frases del texto. Para ello se elabora una anotación manual de las frases que constituyen el set de datos y luego, se generan un set de features para estas que pasaran a entrenar un clasificador lineal.

En \cite{aman2007identifying} se utilizan texto proveniente de blogs para realizar detección de emociones presentes en las oraciones de estos. Para ello recurren primero a una anotación manual de las mismas y luego a la construcción de features para entrenar distintos modelos supervisados.


Luego, en \cite{pang2008opinion} se hace manifiesto la importancia que ha venido ganando el campo del análisis de sentimiento debido al auge del Internet, tanto para usuarios individuales como para la industria de la publicidad, el mercado financiero y la academia, por lo que se hace un recuento de las distintas técnicas y aplicaciones que son consideradas relevantes por los autores hasta la fecha. 




En \cite{pak2010twitter} se identifica twitter como una plataforma útil para extraer texto de diferentes usuarios sobre distintos temas por lo que se propone realizar un análisis de sentimientos sobre la misma. Para ello se procede  a la identificación de tweets que contengan emoticones felices y tristes, así como tweets provenientes de cuentas de medios de noticias para tweets neutrales. luego se procede a la construcción de features usando n-gramas a partir de las palabras presentes en el tweets  con estos se entrenaron varios clasificadores.



En \cite{o2010tweets} se plantea la pregunta si existe una correlación entre el sentimiento encontrado en twitter y las encuestas de opinión. Para ello, se toma una muestra del de mil millones de tweets entre 2008 y 2009 y se toman aquellos tweets que contengan palabras claves asociadas a los temas que se están investigando. EL sentimiento se determina a partir de la proporción de palabras con asociación negativa o positiva presentes en el tema que se esta analizando en un día en particular. Los resultados dan una correlación alta entre el sentimiento encontrado a través del texto y las encuestas.

En \cite{davidov2010enhanced} se parte del supuesto de que los hashtags y los emoticones contienen información relevante el cuanto al sentimiento del tweet, por lo que se hace una selección de 50 hashtags que tengan una asociación fuerte al sentimiento y se entrena un modelo supervisado a partir de los tweets que contengan estos hashtags, construyendo un vector de features para cada uno. El modelo es posteriormente empleado para clasificar otros tweets y jueces humanos verifican su eficacia.


En  \cite{wang2012harnessing} se entrena un modelo de supervisado distante de emociones presentes en twitter. Para ellos, se utilizan hashtags que contengan terminos claves provenientes de las 5 emociones basicas de propuestas por \cite{ekman1993facial} para realizar el llamado de la api. A partir de ahi se entrena el modelo a partir de los features construidos para el texto de los tweets. 

En \cite{roberts2012empatweet} se identifica la necesidad de contar con un corpus que sirva de base para la tarea de identificación de emociones en twitter. Para ello, se seleccionan  14 temas que para los autores tienen un fuerte contenido emocional y las palabras clave asociados a estos para ser usados como hashtags en las extracción. A partir de ahí, manualmente se etiquetaron los tweets con su respectiva emoción. Esto sirvió de base para entrenar un modelo de aprendizaje supervisado.



En \cite{bollen2011modeling} se plantea que a través del uso de twitter, los usurarios reflejan sus estados emocionales, ya sea de una manera explicita al indicar su emoción o de una manera implícita al hablar sobre un tema de interés general. En ese contexto, se genera procede a realizar una medida del estado emocional de una muestra de tweets entre agosto y diciembre del 2008, en donde el mismo se mide a través de ka similaridad presente entre las palabras de los tweets y ciertos términos claves asociados a estados emocionales. Esto permite encontrar que determinados eventos relevantes generan un impacto emocional significativo y durante un periodo de tiempo en los usuarios.

En \cite{tumasjan2010predicting} se plante el uso de twitter como plataforma de medición de la sensación política durante las elecciones parlamentarias en Alemania en el 2009. Una de sus preguntas de investigación estuvo relacionada con los sentimientos que se reflejan en los tweets que mencionan a los políticos que hacen parte de las campañas y para esto, se empleo sobre el texto proveniente de twitter, un software capaz de identificar palabras claves asociadas a estados emocionales y cognitivos en el texto. El resultado fue un perfil emocional para cada político que ve de acuerdo con su discurso político.

En \cite{ceron2016sentiment} se pretende realizar un análisis de sentimientos sobre tweets relacionados con las elecciones presidenciales en Colombia en el 2014 y comparar los resultados de este con las encuestas de opinión. Para ellos, inicialmente obtiene tweets que tengan  palabras claves y hashtags relacionados con las elecciones, luego hace un filtrado de spam y finalmente entrena un modelo supervisado con los features que genera para los tweets restantes. Los resultados no son consistentes por lo que se plantea en futuros trabajos  una caracterización demográfica.



Las RNN (redes neuronales recurrentes) son un tipo de arquitectura de red neuronal cuyo uso es especial para datos secuenciales, tales como las tareas de NLP. En estas, para un ejemplo nuevo, es posible utilizar el resultado del procesamiento de un dato anterior, para el procesamiento del siguiente dato (como en una secuencia de caracteres por ejemplo), es decir la misma topología de pesos recibirá para cada ejemplo nuevo, el resultado del ejemplo anterior ademas del ejemplo nuevo de entrada. Sin embargo, debido a su arquitectura, la optimizacion de los pesos que constituyen la topología se vuelve compleja pues para el calculo del gradiente del error, se deberá pasar tantas veces por los pesos de la red como pasos en el tiempo haya (como numero de caracteres en una palabra) haciendo que en secuencias particularmente largas, este calculo crezca o disminuya en demasía. Para enfrentar este problema , \cite{hochreiter1997long} y \cite{chung2014empirical} plantean nuevas arquitecturas de RNN conocidas como LSTM (Long-Short term memoryzz) y GRU (Gated Recurrent Unit) respectivamente, en donde a través de nuevas unidades que permiten la activación/cancelación de las señales que constituyen la red, se puede realizar la optimizacion de manera directa sin pasar por los pesos de la red.


En general ,las arquitecturas de RNN, al tener una estructura secuencial, se impide la paralelizacion de su computo, pues se necesita las salidas de ejemplos anteriores para llevar a cabo el siguiente paso en el tiempo. Ademas, debido a esta mismo funcionamiento secuencial, si existe un gran numero de pasos en el tiempo, es poco probable que la red tenga en cuenta información presente al inicio. Para solucionar estos inconvenientes, \cite{vaswani2017attention} desarrollan los Transformers, que son un tipo de arquitectura en la que todos los datos de entrada, son ingresados a la red de manera simultanea, como todas las palabras en una frase por ejemplo, y a través de una arquitectura que definen como atención, se realiza una transformación de los datos de manera que la representación de cada dato, en este caso cada palabra, tenga en cuenta que tan importante es la misma para todas las demás palabras de la oración. Esta transformación, aplicada tanto en los datos de entrada como en los de salida, es luego usada para entrenar los pesos de la red. De esta manera la red es capaz de procesar datos en simultaneo así como tener en cuenta toda la información disponible.
 

En \cite{devlin2018bert} se desarrolla BERT

En \cite{acheampong2021transformer} se hace un recuento de el uso de transformers para detectar emociones

EN \cite{canete2020spanish} se propone una aplicación de BERT para español

En \cite{gonzalez2021twilbert}, \cite{huang2019ana} se utiliza bert en twitter para  detectar emociones

En \cite{plaza2020improved}, \cite{gil2013combining} se hace una clasificación de emociones en español.


En \cite{sidorov2012empirical} propone un léxico de palabras en español asociadas a emociones
`









\chapter{Metodología}

\section{Datos}

\subsection{Obtención y análisis exploratorio de datos}

El dataset obtenido esta inicialmente constituido por 585001 tweets, recolectados entre el 22 de mayo y el 22 de junio del 2022, periodo dentro del cual ocurrieron la primera y la segunda vuelta de las elecciones presidenciales en Colombia , el 29 de mayo y 19 de junio respectivamente. Para la extracción de los datos, se utilizaron 173 hashtags con contenido político que tuvieron lugar durante este periodo. Este dataset fue filtrado para remover aquellos tweets que tuvieran menos de 5 palabras, aquellos que tuvieran una proporción de menciones o hashtags mayor al 20\% del total del texto y aquellos que tuvieran links o que provinieran de usuarios con un numero atípico de posteos. Esto redujo la base a 193348 tweets.  Los hashtags utilizados fueron etiquetados en uno de tres sectores políticos: Izquierda, Derecha y Neutro, dependiendo del contenido asociado a dichos hashtags y su tendencia política. En la tabla \ref{table:hashtags} se muestran los hashtags,así como su etiqueta política y cuantos tweets hubo para cada hashtags. En adelante durante el presente trabajo, cuando se hable de sector político, se estará haciendo referencia a esta clasificación.

Cabe resaltar que la asignación de un sector político en particular a un hashtag, fue llevado a cabo a partir de la tendencia política observada en los tweets, lo que no excluye sin embargo la presencia de tweets cuya tendencia política sea contraria. En la tabla \ref{table:ejemplos_1} se pueden encontrar algunos ejemplos de tweets que muestran la relación del hashtag con la orientación política.  El hashtag \#EstallidoSocialEs por ejemplo, fue catalogado como de derecha por presentar en general tweets como el tweet 1 . Sin embargo, hay tweets que lo utilizan, en donde se aprecia un apoyo al candidato de izquierda como el tweet 2 . Del mismo modo el hashtag \#YaEsSuficiente es utilizado en general como apoyo a la izquierda, como en el tweet 3. Sin embargo, también hay casos en donde se usa como apoyo un candidato de derecha como en el ejemplo 4.

\begin{table}
\caption{Ejemplos de tweets con respecto a orientacion politica}
\label{table:ejemplos_1}
\begin{tabular}{{ | p{2cm} | p{13cm} |}}
\toprule
Numero de Ejemplo & Tweet \\
\midrule
1 & @lcvelez @lafm \#EstallidoSocialEs el arma de terror de Petro para obligar a votar por él. \\
2 & Colombia va por el cambio, a redoblar esfuerzos estos 4 días para derrotar a la corrupción. \#PetroYFranciaSonElCambio \#EstallidoSocialEs \\
3 & \#YaEsSuficiente Que los medios proclives al gbno, traten de darle aire boca a boca a un moribundo electoral, FICO. Ante su estancamiento en las encuestas y la distancia que le ha tomado Petro, pretenden en 1 acto de desesperacion, el insuflarlo de votantes de los cuales carece. \\
4 & \#YaEsSuficiente de mentir sobre @ingrodolfohdez , vayan a bucaramanga, vean lo que hizo y ahí si hablen.. \\
5 & \#LoPeorDeEstasElecciones es la división de la gente de este país esta gente y cosas ya que pasan \\
6 & @Zuletalleras Son 4 años O es la. Derecha va realizar un golpe de estado? No cree a sus comentarios son ofensivos e incendiarios.... \#PetroEsPresidente \\
\bottomrule
\end{tabular}
\end{table}

 

La distribución de los hashtags a través de los sectores políticos se puede evidenciar en el gráfico \ref{figure:tweets_cantidad_hashtags} donde se muestra que el sector neutro tiene mas de un  40\% del total de los hashtags, mientras que la izquierda y la derecha, tienen una cantidad semejante, de aproximadamente un 28\% cada uno.


\begin{figure}[t]
	\centering
	\includegraphics{Images & Logos/EDA/Cantidad de Hashtags por sector politico.png} 
	\caption{Porcentaje de Hashtags clasificados según sector político}
	\label{figure:tweets_cantidad_hashtags}
\end{figure}

De manera similar, el gráfico \ref{figure:tweets_porcentaje} muestra la distribución de tweets a lo largo de los sectores políticos. Allí se aprecia que el sector neutro tiene la mayoría de los tweets, con mas del 46\%, luego se encuentra la izquierda con un 33\% y finalmente la derecha con cerca de un 29\%,



\begin{figure}[t]
	\centering
	\includegraphics{Images & Logos/EDA/Porcentaje de tweets por sector político.png} 
	\caption{Porcentaje de Tweets clasificados según sector político}
	\label{figure:tweets_porcentaje}
\end{figure}


Al analizar la cantidad de tweets por sector político a lo largo del tiempo, se obtienen los resultados observados en el gráfico \ref{figure:tweets_porcentaje_tiempo}, en donde se muestra para cada  sector, que porcentaje del total de los tweets tuvo cada día. Se puede evidenciar que hubo algunas fechas particularmente importantes: el 24 de mayo fue el día de un debate y sobresale el sector neutro, el 29 de mayo fue el día de la primera vuelta y sobresalen los tres, el 9 de junio fue el día en el que salieron a la luz los llamados Petro videos, que fueron unos videos filtrados en donde se ve al equipo de campaña de Petro discutiendo estrategias políticas y sobresale la derecha y las fechas cercanas al 19 de junio que fue la segunda vuelta, sobresalen los tres.




\begin{figure}[t]
	\centering
	\includegraphics{Images & Logos/EDA/Porcentaje de tweets por sector a lo largo del tiempo.png} 
	\caption{Porcentaje de tweets clasificados según sector político a lo largo del tiempo}
	\label{figure:tweets_porcentaje_tiempo}
\end{figure}



\subsection{Etiquetado}

Para proporcionar un dataset que sirva de base al eventual entrenamiento del modelo , se procedió al etiquetado manual de 1200 tweets escogidos mediante una muestra aleatoria estratificada proporcional a la cantidad de tweets en los hashtags. Las etiquetas a elegir fueron las emociones presentes en el tweet. 

Una emoción es un patrón de reacción complejo, que involucra elementos experienciales, conductuales y fisiológicos, mediante el cual un individuo intenta lidiar con un asunto o evento personalmente significativo. La cualidad específica de la emoción (por ejemplo, miedo, ira) está determinada por el significado específico que el individuo asigna al evento. Por ejemplo, si el evento implica amenaza, es probable que se genere miedo. Están comprendidas de tres componentes distintos: una experiencia subjetiva, una respuesta fisiológica y una respuesta conductual o expresiva.
Para el contexto del presente trabajo, la experiencia subjetiva y la respuesta fisiológica, no son componentes accesibles. Es así como la atención durante el etiquetado recayó en la respuesta expresiva, concretamente, como el autor del tweet, expresa a través del texto, la respuesta emocional generada en respuesta hacia la entidad en particular a la cual se dirige el tweet. 

LA tarea de etiquetado concretamente consistió en etiquetar un tweet a la vez dentro de una plataforma de etiquetado, en donde se tuvo la posibilidad, mediante un esquema de selección múltiple, de asignar una o varias emociones al tweet, basado en los lineamientos presentes en el manual de etiquetado \footnote{\url{https://docs.google.com/document/d/1hoUYKMaYHSeGeOQ2FqRVyahTin6T09Mu8wtl_HY62O0/edit?usp=sharing}}. Esta tarea fue llevado a cabo por el autor y los directores, quedando así 3 etiquetas independiente para cada uno de los tweets.

Para el esquema de etiquetado, se uso como referencia lo elaborado por \cite{mohammad2015sentiment}, en donde el etiquetador responde varias preguntas, y al ser preguntado por la emoción presente en el tweet,  puede escoger 19 emociones distintas. En el presente trabajo a través de una prueba iterativa para elegir las etiquetas mas adecuadas, se llego a escoger 14 emociones posibles ademas de la categoría otra: Alegria, Agrado, Confianza, Admiración, Miedo, Incertidumbre, Sorpresa, Asombro, Tristeza, Decepción, Asco, Desagrado, Ira, Odio, Otra. En la tabla\ref{table:emotions_description} se presenta una descripción de las emociones usadas.

\begin{longtable}{{ | l | p{13cm} |}}
\caption{Descripción de las emociones usadas} \label{table:emotions_description} \\
\toprule
Emoción & Descripción \\
\midrule
\endfirsthead
\caption[]{Descripción de las emociones usadas} \\
\toprule
Emoción & Descripción \\
\midrule
\endhead
\midrule
\multicolumn{2}{r}{Continued on next page} \\
\midrule
\endfoot
\bottomrule
\endlastfoot
Admiración & La admiración es una emoción social que se siente al observar a personas de competencia, talento o habilidad que superan los estándares. La admiración facilita el aprendizaje social en grupos. La admiración motiva la superación personal a través del aprendizaje de los modelos a seguir. \\
Agrado & Sensación moderada de felicidad o placer que siente una persona por algo que le gusta. \\
Confianza & La confianza implica que una parte se vuelve vulnerable ante otra, asumiendo que esta actuará en su beneficio.En una relación de confianza, el que confía no controla las acciones del otro. \\
Alegría & Es una emoción positiva que suele ir acompañada de bienestar. Se genera como resultado de un evento positivo. \\
Incertidumbre & La incertidumbre es la falta de seguridad, de confianza o de certeza sobre algo. Aparece en situaciones en las que no tenemos control total, en las que nos faltan respuestas e información, y nos puede generar inquietud, inseguridad, estrés, ansiedad e incluso miedo \\
Miedo & El miedo surge ante amenazas reales o imaginarias de daño físico, emocional o psicológico. En textos, se muestra como amenazas hacia el autor del mensaje o lo que se menciona, cuando está vulnerable o en desventaja.  \\
Asombro & La condición de estar asombrado; un estado de asombro abrumador, como por sorpresa o miedo repentino, horror o admiración.  \\
Sorpresa & Se define como una reacción provocada por algo inesperado, extraño o novedoso para la persona. En el texto está principalmente asociada a resultados inesperados o descubrimientos singulares respecto al enunciado del tweet. \\
Decepción & La decepción es la insatisfacción que sigue al fracaso de expectativas o esperanzas. A diferencia del arrepentimiento que se centra en elecciones personales, la decepción se enfoca en el resultado en sí. Puede generar estrés psicológico. \\
Tristeza & La tristeza es un dolor emocional causado por decadencia espiritual, manifestándose en llanto, abatimiento, falta de apetito, cansancio, etc. Ocurre cuando las expectativas no se cumplen o las circunstancias son dolorosas.  \\
Desagrado & Una actitud o un sentimiento de disgusto o aversión. \\
Asco & Contiene una serie de estados con intensidades variables que van desde una leve aprensión hasta una intensa repulsión. Todos los estados de asco se desencadenan por la sensación de que algo es aversivo, repulsivo y/o tóxico.  \\
Odio & El odio es una intensa respuesta emocional negativa hacia ciertas personas, cosas o ideas, generalmente relacionadas con la oposición o repulsión hacia algo. El odio a menudo se asocia con intensos sentimientos de ira, desprecio y disgusto. \\
Ira & La ira surge por objetivos no alcanzados o trato injusto, pudiendo ser peligrosa y relacionada con la violencia. En el texto, el autor reta o reclama por un derecho vulnerado, buscando justicia. \\
\end{longtable}



Este etiquetado se llevo a cabo usando la plataforma web Label Studio. La interfaz de etiquetado se puede observar en la figura \ref{figure:interfaz}.


\begin{figure}[t]
	\centering
	\includegraphics[scale=0.45]{Images & Logos/interfaz.png}
	\caption{Interfaz de etiquetado} 
	\label{figure:interfaz}
\end{figure}

A partir de los resultados obtenidos, se obtuvo el gráfico \ref{figure:correlacion_emociones}, en donde  se calcula la correlacion que tuvieron las etiquetadores en las distintas etiquetas. 

\begin{figure}[t]
	\centering
	\includegraphics[scale=0.65]{Images & Logos/EDA/emotions_correlations.png} 
	\caption{Porcentaje de coorelación entre etiquetadores de las emociones asignadas a los tweets'}
	\label{figure:correlacion_emociones}
\end{figure}


Basándose en estos resultados, se construyeron 4 etiquetas agrupadoras, a partir de la correlacion que presentaron. Estas etiquetas fueron: Alegría que contiene las etiquetas Alegria, Agrado, Confianza y Admiración; Miedo que contiene las etiquetas Miedo e  Incertidumbre; Tristeza que contiene las etiquetas Tristeza y  Decepción; y 
Asco que contiene las etiquetas Asco, Desagrado, Ira y Odio. A cada tweet se le asigno una o varias de estas etiquetas finales, si al menos dos de los etiquetadores coincidían en dichas etiquetas. Estas fueron las etiquetas con las que se entreno el modelo.

El acuerdo observado entre los etiquetadores para cada una de estas 4 emociones, fue medido utilizando el indice de Fleiss Kappa. Los resultados se encuentran en el cuadro \ref{table:agreements}.


\begin{table}[t]
\centering
\begin{tabular}{lllll}
\toprule
 & alegria & miedo & tristeza & asco \\
\midrule
Cantidad de Tweets & 464 & 98 & 103 & 580 \\
Indice de Fleiss $\kappa$ & 0.69 & 0.47 & 0.4 & 0.62 \\
\bottomrule
\end{tabular}
\caption{Cantidad de tweets etiquetados e indice de Fleiss Kappa en cada emoción}
\label{table:agreements}
\end{table}


Se puede apreciar como alegría y asco tuvieron puntajes relativamente altos en comparación con tristeza y miedo. Esto se explica en parte por la cercanía que presentan estas dos ultimas a ascos, como se puede apreciar en el tweet numero 5 de la tabla \ref{table:ejemplos_1}. En el, dos de los etiquetadores coincidieron en asignar tristeza mientras el tercero asigno asco. Este mismo fenómeno ocurren con miedo como se aprecia en el tweet numero 6 de la tabla. Allí también fue el caso en donde uno de los etiquetadores asigno asco  mientras los otros dos asignaron miedo.


\section{Modelos}

\subsection{Modelos pre entrenados}

Los modelos de lenguaje pre entrenados se encuentran disponibles para su uso en el sitio Hugging Face \footnote{\url{https://huggingface.co/}}. Desde allí, basándose en la relevancia del modelo para la tarea del presente trabajo, se escogieron 6 modelos diferentes para ser probados y escoger aquel que tuviera el mejor desempeño.




\begin{table}[!htbp]
\scriptsize
\begin{tabular}{{ | l | p{13cm} |}}
\hline
Modelo & Descripcion  \\
\hline
robertuito  & Modelo de lenguaje preentrenado para contenido generado por usuarios en español, entrenado siguiendo las pautas de RoBERTa en 500 millones de tweets. \\
bertin & Modelo basados en RoBERTa entrenados desde cero en la parte española de mC4 usando Flax \\
electricidad & Modelo base tipo Electra entrenado en el corpus BETO \\
beto & BETO es un modelo BERT entrenado con un gran corpus español. \\
roberta & Modelo base de RoBERTa y ha sido preentrenado utilizando un corpus en español de 570 GB de texto limpio  \\
twitter-xlm & Modelo basado en XLM-roBERTa multilingüe entrenado en ~198 millones de tweets y ajustado para el análisis de sentimientos. \\
\hline
\end{tabular}
\caption{Descripción de los modelos usados}
\label{table:model_description}
\end{table}


Los modelos escogidos fueron robertuito \cite{perez-etal-2022-robertuito}, bertin \cite{BERTIN},  BETO \cite{CaneteCFP2020}, electricidad \cite{mromero2020electricidad-base-discriminator},  roberta \cite{ROBERTA} y twitter-xlm \cite{barbieri2022xlm}. En el cuadro \ref{table:model_description} se encuentra una pequeña descripción de estos así como su localización dentro del hub de Hugging Face.




\subsection{Entrenamiento de los modelos}


Debido a que el fine tuning de estos modelos requiere un poder de computo importante, se decidió utilizar el servicio de Google Colab \footnote{\url{https://colab.research.google.com/}}, en donde se pueden desarrollar notebooks utilizando GPU de manera gratuita. Allí, se utilizo la librería Transformers de Huggingface para poder acceder a los modelos. 

Para poder entrenar los modelos con esta librería, existen dos objetos importantes, el Tokenizer y el trainer. El primero, realiza una transformación de los datos de entrada en vectores tokenizados que puedan ser interpretados por el modelo. EL segundo, recibe como entrada estos datos tokenizados, el modelo a utilizar, los hiperparametros y la métrica a utilizar para evaluar el modelo durante el entrenamiento. Tanto para la definición del tokenizer como la del modelo, se debe especificar la ruta del modelo en particular que se desea implementar.

Los modelos se entrenaron usando los hiperparametros por defecto disponibles en Hugging Face. Estos son principalmente Adamw como algoritmo de optimizacion, con un learning rate de 5e-05, un
decay de 0.0, 3 como el numero de epochs en train y un batch size por nucleo de 8. Estos pueden consultarse en el sitio web \footnote{\url{https://huggingface.co/docs/transformers/v4.30.0/en/main_classes/trainer#transformers.TrainingArguments}}

\subsection{Evaluación de los modelos}

Los modelos fueron entrenados y evaluados utilizando la métrica Micro F1 score definida de la siguiente manera:


\[
{\text{Micro F1}}=\frac{\text{TP} }{\text{TP} + \frac{1}{2}\text{(FP +FN)}}
\]

En donde TP es el numero de verdaderos positivos, FP el numero de falsos positivos y FN el numero de falsos negativos a lo largo de todas las clases. Se eligió esta métrica debido a que el modelo en cuestión permite la clasificación múltiple y con esta métrica se puede evaluar de manera simultanea el desempeño de las distintas clases.

Para entrenar y evaluar los modelos, los datos etiquetados se dividieron en train y test. Se opto por no realizar una tercera partición debido a que no se realiza una hiperoptimizacion de los modelos, sumado al tamaño del dataset etiquetado. Una vez particionados los datos, se procedió a entrenar cada uno de los 6 modelos, 10 veces y a guardar el resultado de su desempeño, con el objetivo de obtener una lectura mas robusta, dada la naturaleza aleatoria de las redes neuronales. El resultado del entrenamiento de cada modelo fue registrado en el sitio Wandb \footnote{\url{https://wandb.ai/}}, cuyo propósito es servir de herramienta de acompañamiento al entrenar modelos.








% You can either break down your chapter into multiple tex files
% or keep in one file and quickly move between sections and subsections using
% 'the File Outline' box at the bottom left.

\nocite{*}
\bibliographystyle{plain}
\bibliography{ref.bib}

\end{document}

