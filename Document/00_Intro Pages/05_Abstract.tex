% -------------------- Abstract page -----------------------%
\begin{center}
{\Large{\bf{Resumen}}}
\end{center}


El estudio de Twitter como medio para el análisis de fenómenos sociales mediante técnicas de procesamiento del lenguaje natural (NLP) es de gran interés. Dentro de estas técnicas, la detección de emociones en el texto es especialmente relevante , ya que permite identificar la respuesta subjetiva de las personas ante los eventos sociales. Los modelos de lenguaje basados en redes neuronales, como los Transformers y específicamente BERT, han reemplazado las técnicas tradicionales de NLP debido a su capacidad para capturar el contexto y las relaciones entre las palabras en el texto. Sin embargo, en español, hay pocos estudios de detección de emociones en redes sociales y ninguno que utilice modelos de lenguaje basados en Transformers en un contexto político. Este trabajo tiene como objetivo emplear modelos basados en BERT para detectar emociones en Twitter durante las elecciones presidenciales de Colombia en 2022. Se utilizará una base de datos de tweets descargados a través de la API, etiquetados manualmente con emociones y se realizará un fine tuning de varios modelos de lenguaje preentrenados para seleccionar el mejor. Este modelo se utilizará para clasificar toda la base de datos y analizar la respuesta emocional de los tweets asociados a sectores políticos y su variación temporal.

\vspace{4cm} % Reduce if text overflowing to a new page. Don't make it too long.
\textbf{Palabras Clave:} [BERT, Colombia, Elecciones, Detección de emociones]

\clearpage
